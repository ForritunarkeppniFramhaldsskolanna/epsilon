
<img src="assets/hello.jpg" alt="music" class="img-polaroid" style="float:right;width:30%;margin:6px;" />

Brynsteinn vinur þinn er einstaklega hrifinn af tónlist, og er búinn að koma
sér upp stóru safni á tölvunni sinni. Því fylgir þó sá galli að erfitt getur
reynst að finna ákveðna tónlist í safninu. Brynsteinn leitar því til þín og þú
kemur með hugmynd að forriti þar sem leitað er að tónlist með
*fyrirspurnarstrengjum*.

Fyrirpurnarstrengur er strengur sem inniheldur hvaða staf sem löglegur er
í skráarheiti, t.d. bókstafi, tölur, punkt, kommu, bil, o.s.frv. Þar að auki getur
fyrirspurnarstrengur innihaldið í mesta lagi eina stjörnu, `*`.
Fyrirspurnarstrengur er sniðmát af skráarnafni þar sem stjarnan táknar hvaða
streng sem er (þ.m.t. tóma strenginn). Ef skráarnafn fellur að sniðmátinu, þá
segjum við að að skráarnafnið samsvari fyrirspurnarstrengnum. T.d. samsvara
allar skrár sem enda á `.mp3` fyrirspurnarstrengnum `*.mp3`.

Fyrsta lína inntaksins inniheldur fyrirspurnarstrenginn $Q$. Næsta lína
inniheldur eina heiltölu $n > 0$, og næstu $n$ línur innihalda skráarnöfn.
Fyrir sérhvert skráarnafn á að skrifa út `Passar` ef skráarnafnið samsvarar
$Q$, annars skal skrifa út `Passar ekki`.
