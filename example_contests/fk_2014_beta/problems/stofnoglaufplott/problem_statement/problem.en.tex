
Í tölfræði er oft verið að vinna með mikið af gögnum. Þess vegna getur verið
mikilvægt að setja gögnin fram á formi sem einfaldar úrvinnslu. Þegar verið er
að vinna með lista af heiltölum, þá kallast eitt af þessum formum Stofn-og-lauf
rit. Segjum að við séum með tveggja-stafa heiltölurnar

<div style="text-align:center"><p class="tex2jax_process">$10, 12, 12, 13, 28, 45, 46, 47, 49, 49, 93$</p></div>

Þá lítur Stofn-og-lauf rit fyrir tölurnar svona út

    1 0223
    2 8
    4 56799
    9 3

Aftari tölustafurinn í tölunum er settur í röð í línuna sem samsvarar fremri
tölustafnum. Athugið að línurnar eru raðaðar eftir fremsta tölustafnum, og
runan af tölustöfum í hverri línu er röðuð.

Í þessu verkefni á að útfæra forrit sem færir lista af tveggja-stafa heiltölum
yfir í Stofn-og-lauf rit.

Inntakið inniheldur fyrst eina línu með heiltölunni $n$. Þar eftir fylgir önnur
lína með $n$ tveggja-stafa heiltölum aðskildum með bili. Úttakið á að innihalda
tölurnar á Stofn-og-lauf rit formi.

