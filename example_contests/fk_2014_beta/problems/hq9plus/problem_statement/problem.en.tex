\problemname{HQ9+}

% \includegraphics[scale=0.4]{umad.gif}

Til er ógrynni af forritunarmálum sem flokkast undir það að vera sérstök(\textit{esoteric}). Þau eru nánast eingöngu til gamans gerð, en
oft er erfitt að skrifa hefðbundin forrit í svoleiðis forritunarmálum, og svo
ekki sé minnst á að forritskóðinn lítur yfirleitt alltaf mjög furðulega út.

Dæmi um nokkur slík forritunarmál eru eftirfarandi:

\begin{enumerate}
    \item \textbf{Brainf***}: Hefur aðeins fjórar aðgerðir til að vinna með minni:
    <, >, +, -, tvær aðgerðir til að lesa og skrifa á skjá: ., ,, og eina gerð
    af lykkju: [].
    \item \textbf{Whitespace}: Forritskóðinn samanstendur af mismunandi biltáknum (e.
    whitespace).
    \item \textbf{Chef}: Forritskóði lítur út eins og mataruppskrift.
    \item \textbf{HQ9+}: Hefur nákvæmlega fjórar aðgerðir: H skrifar út "Hello World", Q
    skrifar út forritskóðann sjálfan, 9 skrifar út textann við laginu "Ninty-Nine
    Bottles of Beer on the Wall" og + hækkar teljara um einn. Í byrjun
    forritsins er teljarinn stilltur sem 0.
\end{enumerate}

Skrifið forrit sem keyrir forrit skrifað í forritunarmálinu HQ9+ og skrifar út
gildi teljarans eftir að forritið hefur keyrt.

Inntakið inniheldur eina línu sem inniheldur forritskóðann, skrifaðan í
forritunarmálinu HQ9+. Forritskóðinn inniheldur aðeins táknin H, Q, 9 og +, og er í
mesta lagi 1000 stafir að lengd.
