
<img src="assets/umad.gif" alt="umad" class="img-polaroid" style="float:right;width:30%;margin:6px;" />

Til er ógrynni af forritunarmálum sem flokkast undir það að vera <abbr
title="esoteric">sérstök</abbr>. Þau eru nánast eingöngu til gamans gerð, en
oft er erfitt að skrifa hefðbundin forrit í svoleiðis forritunarmálum, og svo
ekki sé minnst á að forritskóðinn lítur yfirleitt alltaf mjög furðulega út.

Dæmi um nokkur slík forritunarmál eru eftirfarandi:

<ul>
    <li><b>Brainf***</b>: Hefur aðeins fjórar aðgerðir til að vinna með minni:
    <, >, +, -, tvær aðgerðir til að lesa og skrifa á skjá: ., ,, og eina gerð
    af lykkju: [].</li>
    <li><b>Whitespace</b>: Forritskóðinn samanstendur af mismunandi biltáknum (e.
    whitespace).</li>
    <li><b>Chef</b>: Forritskóði lítur út eins og mataruppskrift.</li>
    <li><b>HQ9+</b>: Hefur nákvæmlega fjórar aðgerðir: H skrifar út "Hello World", Q
    skrifar út forritskóðann sjálfan, 9 skrifar út textann við laginu "Ninty-Nine
    Bottles of Beer on the Wall" og + hækkar teljara um einn. Í byrjun
    forritsins er teljarinn stilltur sem 0.</li>
</ul>

Skrifið forrit sem keyrir forrit skrifað í forritunarmálinu HQ9+ og skrifar út
gildi teljarans eftir að forritið hefur keyrt.

Inntakið inniheldur eina línu sem inniheldur forritskóðann, skrifaðan í
forritunarmálinu HQ9+. Forritskóðinn inniheldur aðeins táknin H, Q, 9 og +, og er í
mesta lagi 1000 stafir að lengd.
