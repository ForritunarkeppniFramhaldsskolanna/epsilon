\problemname{Sáldur Eratosthenesar}

Frumtala er jákvæð heiltala sem hefur nákvæmlega tvo deila: töluna sjálfa og
töluna einn. Fyrstu sex frumtölurnar eru $2, 3, 5, 7, 11, 13$ og dæmi um stóra
frumtölu er $123457$. Dæmi um tölur sem eru ekki frumtölur eru $6$ (bæði $2$ og $3$ eru
deilar hennar) og $35$ (bæði $5$ og $7$ eru deilar hennar). Takið eftir að 1 er ekki
frumtala, en hún hefur bara einn deili.

Frumtölur mynda grunninn að talnafræði og eru mikilvægar í nútíma
dulritunarkerfum. Í þessu verkefni ætlum við að skoða reiknirit til að finna allar
frumtölur upp að gefnu efra marki, sem við skulum kalla $n$.

Einfaldasta reikniritið fer í gegnum allar tölur frá $1$ upp í $n$, og athugar
hvort talan sé frumtala. Til eru hraðari reiknirit, og ætlum við að skoða eitt
þeirra. Reikniritið kallast Sáldur Eratosþenesar og virkar
á eftirfarandi hátt:

\begin{enumerate}
    \item Búa til lista af öllum tölum frá $1$ upp í $n$
    \item Krota $X$ yfir töluna $1$ í listanum
    \item Fyrir hverja heiltölu $i$ frá $2$ og þar til $i\times i > n$:
        \begin{enumerate}
            \item Ef búið er að krota $X$ yfir töluna $i$ í listanum, halda áfram með lykkju í skrefi $3$
            \item Fyrir hverja heiltölu $k$ frá $i$ og þar til $k\times i > n$:
                \begin{enumerate}
                    Krota $X$ yfir töluna $k\times i$ í listanum
                \end{enumerate}
        \end{enumerate}
    
\end{enumerate}

Reikniritið útilokar tölur sem eru ekki frumtölur með því að krota $X$ yfir
þær, og í lok reikniritsins munu tölurnar sem ekki er búið að krota yfir vera
allar frumtölur á bilinu $1$ upp í $n$.

Skrifið forrit sem keyrir Sáldur Eratosþenesar. Forritið á auk þess að skrifa
út listann af tölunum í hvert skipti rétt áður en skref $4$ er keyrt. Ef búið
er að krota $X$ yfir tölu á að skrifa út stafinn $X$, en annars á að skrifa út
töluna sjálfa. Stakt bil á að vera á milli talnanna, og á hver listi sem
skrifaður er út að vera í sér línu. Þegar keyrslu reikniritsins er lokið á að
skrifa listann út einu sinni enn.

Inntakið inniheldur eina línu með heiltölunni $1 \leq n < 1000$.
