
Frumtala er jákvæð heiltala sem hefur nákvæmlega tvo deila: töluna sjálfa og
töluna einn. Fyrstu sex frumtölurnar eru $2, 3, 5, 7, 11, 13$ og dæmi um stóra
frumtölu er $123457$. Dæmi um tölur sem eru ekki frumtölur eru $6$ (bæði $2$ og $3$ eru
deilar hennar) og $35$ (bæði $5$ og $7$ eru deilar hennar). Takið eftir að 1 er ekki
frumtala, en hún hefur bara einn deili.

Frumtölur mynda grunninn að talnafræði og eru mikilvægar í nútíma
dulritunarkerfum. Í þessu verkefni ætlum við að skoða reiknirit til að finna allar
frumtölur upp að gefnu efra marki, sem við skulum kalla $n$.

Einfaldasta reikniritið fer í gegnum allar tölur frá $1$ upp í $n$, og athugar
hvort talan sé frumtala. Til eru hraðari reiknirit, og ætlum við að skoða eitt
þeirra. Reikniritið kallast Sáldur Eratosþenesar og virkar
á eftirfarandi hátt:

<ol>
    <li value="1">Búa til lista af öllum tölum frá <span class="tex2jax_process">$1$</span> upp í <span class="tex2jax_process">$n$</span></li>
    <li value="2">Krota <span class="tex2jax_process">$X$</span> yfir töluna <span class="tex2jax_process">$1$</span> í listanum</li>
    <li value="3">Fyrir hverja heiltölu <span class="tex2jax_process">$i$</span> frá <span class="tex2jax_process">$2$</span> og þar til <span class="tex2jax_process">$i\times i > n$</span>:
        <ol>
            <li value="4">Ef búið er að krota <span class="tex2jax_process">$X$</span> yfir töluna <span class="tex2jax_process">$i$</span> í listanum, halda áfram með lykkju í skrefi <span class="tex2jax_process">$3$</span></li>
            <li value="5">Fyrir hverja heiltölu <span class="tex2jax_process">$k$</span> frá <span class="tex2jax_process">$i$</span> og þar til <span class="tex2jax_process">$k\times i > n$</span>:
                <ol>
                    <li value="6">Krota <span class="tex2jax_process">$X$</span> yfir töluna <span class="tex2jax_process">$k\times i$</span> í listanum</li>
                </ol>
            </li>
        </ol>
    </li>
</ol>

Reikniritið útilokar tölur sem eru ekki frumtölur með því að krota $X$ yfir
þær, og í lok reikniritsins munu tölurnar sem ekki er búið að krota yfir vera
allar frumtölur á bilinu $1$ upp í $n$.

Skrifið forrit sem keyrir Sáldur Eratosþenesar. Forritið á auk þess að skrifa
út listann af tölunum í hvert skipti rétt áður en skref $4$ er keyrt. Ef búið
er að krota $X$ yfir tölu á að skrifa út stafinn $X$, en annars á að skrifa út
töluna sjálfa. Stakt bil á að vera á milli talnanna, og á hver listi sem
skrifaður er út að vera í sér línu. Þegar keyrslu reikniritsins er lokið á að
skrifa listann út einu sinni enn.

Inntakið inniheldur eina línu með heiltölunni $1 \leq n < 1000$.
