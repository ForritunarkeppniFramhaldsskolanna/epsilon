
<img src="assets/cable.jpg" alt="cable" class="img-polaroid" style="float:right;width:30%;margin:6px;" />

Verið er að byggja nýjan bæ á austurlandi. Í bænum eru $n$ hús. Nú þarf að
útvega öllum húsunum internet, og er það gert með því að leggja netkapla á
milli húsa. Hús númer $1$ er þegar komið með net, en það er tengt með löngum
netkapli við næsta bæ.

Ykkar verkefni er að finna út hvaða hús vantar nettengingu. Hús er nettengt
ef það hefur netkapal í annað hús sem er nettengt.

Á fyrstu línu inntaksins eru tvær heiltölur $1 \leq n \leq 100$, $0 \leq m \leq
n(n-1)/2$, þar sem $n$ er fjöldi húsa og $m$ er fjöldi netkapla. Svo fylgja $m$
línur, en hver þeirra inniheldur tvær heiltölur $a$ og $b$, sem táknar að hús
$a$ og $b$ eru tengd með netkapli.

Ef öll húsin eru nettengd þá á úttak að innihalda línuna &ldquo;<tt>Allir
nettengdir</tt>&rdquo;. Ef ekki, þá á að skrifa út númer húsanna sem eru ekki
nettengd, eitt á línu, í hækkandi röð.

