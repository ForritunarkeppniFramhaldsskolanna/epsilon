\problemname{Evil and Odious}

\includegraphics[scale=0.4]{masks.jpg}

Það er ekki algengt að lýsingarorð fylgi tölum, en það vill svo til að í
heimi tölvunarfræðinga og stærðfræðinga eru allar tölur annað hvort
"slæmar" (e. evil) eða "illa þokkaðar"
(e. odious).

Slæmum og illa þokkuðum tölum er skipt eftir því hversu margir ásar eru
í táknun þeirra á tvíundarformi. Slæmar tölur hafa sléttan fjölda ása á
meðan illa þokkaðar tölur hafa oddatölu fjölda ása.

Sem dæmi, þá er 5 slæm tala. Hún er táknuð sem $101_2$ á tvíundaformi,
sem hefur sléttan fjölda ása. Þá er 2 illa þokkuð tala. Hún er
táknuð sem $10_2$ á tvíundarformi, sem hefur oddatölu fjölda ása.

Nú eru þetta frekar neikvæð lýsingarorð. Ætli það sé hægt að leggja saman
slæma tölu og illa þokkaða tölu og fá út góða tölu?

Skilgreinum góðar tölur sem runu af tölum þar sem $n$-ta talan í rununni er
summa $n$-tu slæmu tölunnar og $n$-tu illa þokkuðu tölunnar. Til dæmis er
fyrsta góða talan 1. Hún er fengin með að leggja saman 0 (fyrsta slæma talan)
og 1 (fyrsta illa þokkaða talan). Önnur góða talan er 5. Hún er fengin með
að leggja saman 3 (önnur slæma talan) og 2 (önnur illa þokkaða talan).

Skrifið forrit sem skrifar út $n$-tu góðu töluna. Inntakið inniheldur eina
línu með heiltölunni $1 \leq n \leq 10^{16}$.

