\problemname{Orðasamsetning}

\includegraphics[scale=0.4]{words.png}

Gerður litla elskar að setja saman orð. Til dæmis setur hún saman orðin
\textit{"eld"} og \textit{"hús"} og fær þá nýja orðið
\textit{"eldhús"}. Hún hefur núna tekið eftir því að sumar
orðasamsetningar eru skemmtilegri en aðrar. Ein slík orðasamsetning er mynduð
með orðunum \textit{"frum"} og \textit{"umröðun"}. Það sem henni
finnst sniðugt er að fyrra orðið endar á \textit{um} og seinna orðið byrjar á \textit{um}. Í
staðinn fyrir að setja orðin saman og fá \textit{"frumumröðun"}, þá fær
hún \textit{"frumröðun"}.

Inntak inniheldur tvær línur, þar sem fyrri línan inniheldur fyrra orðið og
seinni línan inniheldur seinna orðið.

Úttak á að innihalda eina línu með samsetta orðinu. Ef endir fyrra orðsins og
byrjun seinna orðsins eru eins, þá á að fella þann hluta saman. Ef hægt er að
gera þetta á marga vegu, þá skulið þið láta samsetta orðið vera eins stutt og
mögulegt er.

