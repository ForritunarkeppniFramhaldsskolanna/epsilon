
<img src="assets/words.png" alt="words" class="img-polaroid" style="float:right;width:30%;margin:6px;" />

Gerður litla elskar að setja saman orð. Til dæmis setur hún saman orðin
*&ldquo;eld&rdquo;* og *&ldquo;hús&rdquo;* og fær þá nýja orðið
*&ldquo;eldhús&rdquo;*. Hún hefur núna tekið eftir því að sumar
orðasamsetningar eru skemmtilegri en aðrar. Ein slík orðasamsetning er mynduð
með orðunum *&ldquo;frum&rdquo;* og *&ldquo;umröðun&rdquo;*. Það sem henni
finnst sniðugt er að fyrra orðið endar á *um* og seinna orðið byrjar á *um*. Í
staðinn fyrir að setja orðin saman og fá *&ldquo;frumumröðun&rdquo;*, þá fær
hún *&ldquo;frumröðun&rdquo;*.

Inntak inniheldur tvær línur, þar sem fyrri línan inniheldur fyrra orðið og
seinni línan inniheldur seinna orðið.

Úttak á að innihalda eina línu með samsetta orðinu. Ef endir fyrra orðsins og
byrjun seinna orðsins eru eins, þá á að fella þann hluta saman. Ef hægt er að
gera þetta á marga vegu, þá skulið þið láta samsetta orðið vera eins stutt og
mögulegt er.

